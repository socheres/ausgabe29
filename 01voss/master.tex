\documentclass[a4paper,
fontsize=11pt,
%headings=small,
oneside,
numbers=noperiodatend,
parskip=half-,
bibliography=totoc,
final
]{scrartcl}

\usepackage{synttree}
\usepackage{graphicx}
\setkeys{Gin}{width=.4\textwidth} %default pics size

\graphicspath{{./plots/}}
\usepackage[ngerman]{babel}
\usepackage[T1]{fontenc}
%\usepackage{amsmath}
\usepackage[utf8x]{inputenc}
\usepackage [hyphens]{url}
\usepackage{booktabs} 
\usepackage[left=2.4cm,right=2.4cm,top=2.3cm,bottom=2cm,headheight=25.60228pt,includeheadfoot]{geometry}
\usepackage{eurosym}
\usepackage{multirow}
\usepackage[ngerman]{varioref}
\setcapindent{1em}
\renewcommand{\labelitemi}{--}
\usepackage{paralist}
\usepackage{pdfpages}
\usepackage{lscape}
\usepackage{float}
\usepackage{acronym}
\usepackage{eurosym}
\usepackage[babel]{csquotes}
\usepackage{longtable,lscape}
\usepackage{mathpazo}
\usepackage[flushmargin,ragged]{footmisc} % left align footnote

%%url brekas grrr
\def\UrlBreaks{\do\a\do\b\do\c\do\d\do\e\do\f\do\g\do\h\do\i\do\j\do\k\do\l%
\do\m\do\n\do\o\do\p\do\q\do\r\do\s\do\t\do\u\do\v\do\w\do\x\do\y\do\z\do\0%
\do\1\do\2\do\3\do\4\do\5\do\6\do\7\do\8\do\9\do\-}%

\usepackage{listings}

\urlstyle{same}  % don't use monospace font for urls

\usepackage[fleqn]{amsmath}

%adjust fontsize for part

%% geometry
\clubpenalty = 10000 
\widowpenalty = 10000 
\displaywidowpenalty = 10000
%% tightlist

\providecommand{\tightlist}{%
  \setlength{\itemsep}{0pt}\setlength{\parskip}{0pt}}

\usepackage{sectsty}
\partfont{\large}

%Das BibTeX-Zeichen mit \BibTeX setzen:
\def\symbol#1{\char #1\relax}
\def\bsl{{\tt\symbol{'134}}}
\def\BibTeX{{\rm B\kern-.05em{\sc i\kern-.025em b}\kern-.08em
    T\kern-.1667em\lower.7ex\hbox{E}\kern-.125emX}}

\usepackage{fancyhdr}
\fancyhf{}
\pagestyle{fancyplain}
\fancyhead[R]{\thepage}

%meta

%meta

\fancyhead[L]{J. Voß \\ %author
LIBREAS. Library Ideas, 29 (2016). % journal, issue, volume.
\href{http://nbn-resolving.de/urn:nbn:de:kobv:11-100238115
}{urn:nbn:de:kobv:11-100238115}} % urn
\fancyhead[R]{\thepage} %page number
\fancyfoot[L] {\textit{Creative Commons BY 3.0}} %licence
\fancyfoot[R] {\textit{ISSN: 1860-7950}}

\title{\LARGE{Versuch einer Bibliographie von Bibliographien von Bibliographien von Bibliographien}} %title %title
\author{Jakob Voß} %author

\setcounter{page}{}

\usepackage[colorlinks, linkcolor=black,citecolor=black, urlcolor=blue,
breaklinks= true]{hyperref}

\date{}
\begin{document}

\maketitle
\thispagestyle{fancyplain} 

%abstracts
\begin{abstract}
Der Artikel beschreibt die Erstellung einer Meta-Meta-Meta-Bibliographie
und liefert eine Bewertung des Ergebnis, bestehend aus vier
Universalbibliographien und vier Auswahlbibliographien.
\end{abstract}

%body
\section*{Einleitung}\label{einleitung}

Die Natur von \emph{Bibliographien der Bibliographien}, also
Bibliographien die ihrerseits bibliographische Verzeichnisse auflisten,
wird in den meisten Standardwerken zur Bibliographiekunde erklärt.
Werden mehrere solcher \emph{Metabibliographien} oder
\emph{Bibliographien zweiter Ordnung} in einem Verzeichnis
zusammengefasst, so lässt sich dieses Verzeichnis als
\emph{Bibliographie von Metabibliographien},
\emph{Meta-Metabibliographie}, \emph{Bibliographie dritter Ordnung} oder
\emph{Bibliographien von Bibliographien von Bibliographien} bezeichnen.
Im Folgenden wird der Versuch einer groben Übersicht über solche
Meta-Metabibliographien unternommen.

\section*{Untersuchung}\label{untersuchung}

Im Vergleich zu Meta-Bibliographien ist die Anzahl von
Meta-Metabibliographien klein und das Interesse an dieser Art von
Verzeichnissen gering. Im Wikipedia- Artikel \enquote{Bibliografie} wird
das Konzept (seit einem Eintrag im Februar 2007) lediglich als Idee
erwähnt (Lexoldie, 2007):

\begin{quote}
Es gibt bereits eine ganze Reihe von Metabibliografien zu einzelnen
Fachgebieten, sodass die Zusammenstellung einer Bibliografie der
Metabibliografien durchaus sinnvoll wäre.
\end{quote}

Der Konjunktiv deutet an, dass die Existenz von Meta-Metabibliographien
unsicher ist. Es gibt sie allerdings! So führt eine Google-Suche nach
\enquote{bibliography of bibliographies of bibliographies} unter anderem
zu folgendem Ergebnis (\enquote{A World Bibliography of Bibliographies}
1939):

\begin{quote}
Eventually the ever increasing number of bibliographies of
\textgreater{} bibliographies has justified the publication of
Mr.~Josephson's \textgreater{} bibliography of bibliographies of
bibliographies.
\end{quote}

Verwiesen wird hier auf die zunächst 1901 und in erweiterter Auflage
zwischen 1910 und 1913 herausgegebene Bibliographie von Aksel Josephson.
Neben der Abhandlung von Taylor (1955), der auf Josephson (1901) als
\enquote{first separate publication of such a list} verweist, ist dies
auch schon die einzige ausschließliche Meta-Metabibliographie, die in
Form eines selbständigen Werkes herausgegeben wurde. Taylor (1955, 129)
merkt zudem an, dass die Idee von Josephson nicht neu sei und dass sich
vergleichbare Listen in anderen Nachschlagewerken finden, zum Beispiel
in den Bibliographien von Bibliographien von Peignot (1812) und
Petzholdt (1866).

Abgesehen von Taylor selbst sowie den von ihm angeführten Quellen, blieb
jedoch die Suche nach weiteren umfassenden Bibliographien dritter
Ordnung erfolglos. Es scheint sich auch später niemand mehr intensiver
mit Metabibliographien als Forschungsgegenstand auseinandergesetzt zu
haben, so dass das Vorhandensein weiterer umfangreicher Bibliographien
von Metabibliographien ebenfalls zu bezweifeln ist. In der
englischsprachigen Wikipedia gibt es zwar einen eigenen kurzen Artikel
\enquote{Metabibliography}\footnote{\url{https://en.wikipedia.org/wiki/Metabibliography},
  Stand vom Anfang März 2016}, dieser enthält jedoch lediglich einige
eher willkürlich zusammengestellt Beispiele, so dass diese Liste,
zumindest momentan, nicht als weitere Meta-Metabibliographie in Frage
kommt.

Hjørland (2008) merkt treffend an, dass ihr praktischer Wert für die
Recherche beschränkt ist:

\begin{quote}
Metabibliographies may again be found in meta-meta bibliographies, and
so on. In most real life situations is this a highly problematic way to
seek information. Why? Firstly because bibliographies are included in
other bibliographies. One usually needs not go to
meta-meta-bibliographies. Secondly because the primary literature via
its bibliographical references is a more or less self-organized
bibliographical system, which is often underestimated by
LIS-professionals.
\end{quote}

Dennoch konnten im Rahmen von Recherchehilfen und Fachliteratur zur
Bibliographiekunde einige begrenzte Meta-Metabibliographien ermittelt
werden (Birch, 2015; Gorraiz, 2007?; Keenan, 2015; Rösch \& Härkönen,
2003). Auf eine umfangreichere Recherche wurde allerdings zugunsten
einer einfachen Internetsuche verzichtet, so dass lediglich frei im
Volltext zugängliche Bibliographien berücksichtigt sind.

Neben selbständigen und unselbständigen Bibliographien lassen sich aus
Literaturdatenbanken \enquote{virtuelle Bibliographien} erstellen. So
sollte beispielsweise die Menge aller mit der DDC-Notation
016.016\footnote{Synthetische Notation aus der Grundnotation 016
  (Bibliografien und Kataloge von Werken über einzelne Themen) und der
  Notation 016 (Bibliografien, Kataloge, Indizes) aus Anhängetafel 1.}
erschlossenen Titel eine Liste von Meta-Bibliographien ergeben.
Erschließungsqualität und Recherchemöglichkeit der meisten Kataloge
lassen allerdings so stark zu wünschen übrig, dass auch dieses Verfahren
keine weitere Meta-Metabibliographien ergibt.

\section*{Die Bibliographie}\label{die-bibliographie}

\paragraph{Universalbibliographien}\label{universalbibliographien}

Universalbibliographien umfassen Bibliographien möglichst aller
Metabibliographien:

\begin{itemize}
\item
  Josephson (1901), gedruckt in 500 Exemplaren,\footnote{Online
    digitalisiert verfügbar unter
    \url{https://archive.org/details/bibliographiesof00joserich}}
  enthält 156 Titel aus den Jahren 1664 bis 1900. Die Sammlung bildet
  trotz aller Schwächen die Grundlage der folgenden Werke.
\item
  Grundtvig (1903)\footnote{Online digitalisiert verfügbar unter
    \url{http://www.digizeitschriften.de/dms/resolveppn/?PID=GDZPPN000264148},
    Nachdruck auch bei Frank (1978), S.182-209.} korrigiert und ergänzt
  Josephson im Rahmen einer allgemeinen Kritik verschiedener
  Metabibliographien.
\item
  Zwischen 1911 und 1913 gab Josephson (1911a, 1911b, 1912, 19121913b,
  19121913a) eine zweite Auflage seiner Bibliographie heraus, bei der er
  einen Teil der Kritik von Grundtvig berücksichtigte. Wie Taylor (1955,
  130) anmerkt, bleibt das Werk jedoch hinter seinen Ambitionen
  zurück.\footnote{Es bleibt offen, in wie weit Josephson von Paul
    Otlet's zeitgleich agierendem \emph{Répertoire Bibliographique
    Universel} beeinflusst wurde. Sein Vorschlag zur Einrichtung eines
    Bibliographischen Instituts (Josephson, 1905) weist zumindest starke
    Parallelen auf.} Als Mitarbeiter an der zweiten Auflage werden
  Charles Henry Lincoln, Adolf C. von Noé und Selma Nachman genannt.
\item
  Taylor (1955)\footnote{Online digitalisiert verfügbar unter
    \url{https://archive.org/details/AHistoryOfBibliographiesOfBibliographies}}
  stellt ausführlich die Geschichte der allgemeinen Metabibliographien
  seit 392 v. Chr. dar. Die Abhandlung enthält Beschreibungen und
  Bewertungen der aufgeführten Werke und kann als (einziges)
  Standardwerk zum Thema angesehen werden.
\end{itemize}

\paragraph{Auswahlbibliographien}\label{auswahlbibliographien}

Auswahlbibliographien umfassen Empfehlungslisten von Metabibliographien
sowie Bibliographien von Metabibliographien eines bestimmten
Fachgebietes:

\begin{itemize}
\item
  Rösch \& Härkönen (2003) behandeln \enquote{Bibliographien der
  Bibliographien} im Rahmen einer Lehrveranstaltung zu allgemeinen
  Informationsmitteln. Die Sammlung enthält zwei Lehrbücher, vier
  retrospektive Verzeichnisse, zwei laufende Verzeichnisse und vier
  Verzeichnisse von allgemeinen Informationsmitteln.
\item
  Gorraiz (2007?) führt in seiner \enquote{Bibliographie der
  Bibliographien} zwei Lehrbücher, acht retrospektive Verzeichnisse und
  fünf laufende Verzeichnisse auf zwischen 1945 und 2003 an. Zur
  Einführung verweist er auf Taylor (1955). Die Bibliographie ist Teil
  eines Leitfaden zur Bibliographiekunde.
\item
  Die von Keenan (2015) herausgegebene Liste von \enquote{Bibliographies
  of Bibliographies} beinhaltet insgesamt 19 Bibliographien von
  Bibliographien und anderen Nachschlagewerken zum Thema Russland,
  Osteuropa und Eurasien. Sie ist Teil eines von der Princeton
  University Library herausgegebenen Rechercheführers für diese Gebiete.
\item
  Birch (2015) listet in seinem umfangreichen Nachschlagewerk von
  Philatelie- Quellen als \enquote{Bibliography of Bibliographies of
  Bibliographies} zwei Meta- Metabibliographien auf, von denen
  allerdings eine nur in geringer Auflage an ausgewählte Personen
  verteilt wurde.
\end{itemize}

\paragraph{Daten der Bibliographie}\label{daten-der-bibliographie}

Die angemessene Form der Erfassung von Literatur ist eigentlich eine
Datenbank. Die Einträge dieser Bibliographie liegen zumindest im
BibTeX-Format vor.\footnote{Unter
  \url{https://github.com/jakobib/libreas2016b/blob/master/bibliography.bib}}
Da dieses Format keine Normdaten unterstützt, hier die Identifier der
Werke und beteiligten Personen (sofern vorhanden):

\subparagraph{Werke}\label{werke}

\begin{itemize}
\tightlist
\item
  Josephson (1901)
  \url{http://worldcat.org/entity/work/id/4932415}\footnote{Die von OCLC
    herausgegebenen Werk-URIs lassen sich wie unter
    \textless{}http://zbw.eu/labs/de/blog
    /other-editions-of-this-work-an-experiment-with-oclcs-lod-work-identifiers\textgreater{}
    beschrieben abrufen.}
\item
  Josephson (1911ff) \url{http://worldcat.org/entity/work/id/2242187655}
\item
  Grundtvig (1903) \url{http://worldcat.org/entity/work/id/1808537611}
\item
  Taylor (1955) \url{http://worldcat.org/entity/work/id/1711838}
\end{itemize}

\subparagraph{Personen}\label{personen}

\begin{itemize}
\tightlist
\item
  Vilhelm Grundtvig (1866-1950) \url{http://viaf.org/viaf/47127585}
\item
  Archer Taylor (1890-1973) \url{http://viaf.org/viaf/49251338}
\item
  Aksel Gustav Salomon Josephson (1860-1944)
  \url{http://viaf.org/viaf/44667288}
\item
  Selma Nachman \url{http://viaf.org/viaf/53909744}
\item
  Charles Henry Lincoln (1869-1938) \url{http://viaf.org/viaf/65064279}
\item
  Adolf Carl von Noé (1873-1939) \url{http://viaf.org/viaf/160187725}
\item
  Hermann Rösch (1954-) \url{http://viaf.org/viaf/142145067343066631382}
\item
  Sonja Härkönen \url{http://viaf.org/viaf/62491635}
\end{itemize}

\section*{Zusammenfassung und
Bewertung}\label{zusammenfassung-und-bewertung}

Der vorliegende Versuch einer Bibliographie von Bibliographien von
Bibliographien von Bibliographien dient in erster Linie dazu, eine
Fachveröffentlichung mit ausgefallenem Titel zu lancieren. Da es sich
schon bei Bibliographien um Dokumente über Dokumente, also um
\emph{Meta-Dokumente} handelt, bildet diese Übersicht von
Meta-Metabibliographien quasi ein \emph{Meta-
Meta-Meta-Meta-Dokument}.\footnote{Also ein \(Meta^{2^2}\)-Dokument.
  Diese Bezeichnung wäre allerdings ebenso verwirrend wie albern.}

Das Ergebnis ist überschaubar und vermutlich nicht vollständig:
Insgesamt konnten lediglich vier allgemeine Bibliographien von
Metabibliographien ausfindig gemacht werden, bei Zusammenfassung
verschiedener Auflagen sogar nur drei. Darüber hinaus sind vier
Auswahlbibliographien aufgeführt, die nicht auf Vollständigkeit angelegt
sind und/oder sich auf ein bestimmtes Fachgebiet beschränken. Es ist zu
vermuten, dass im Rahmen von allgemeineren Bibliographien,
Nachschlagewerken und Recherchehilfen weitere, derart begrenzte
Meta-Meta\-biblio\-graphien (oder zumindest einfache Literaturlisten)
existieren.

Trotz dieser Schwächen kann, in Ermangelung ähnlicher Unterfangen, die
vorliegende Übersicht als bislang umfangreichstes Verzeichnis von
Meta-Meta- Bibliographien angesehen werden. Sollten weitere solche
Bibliographien existieren oder erstellt werden, so ist mit einer
Meta-Meta-Meta-Bibliographie zu rechnen. Forschungsbedarf besteht vor
allem zur Entwicklung von Metabibliographien seit 1955, was mit der
Einführung von bibliographischen Datenbanken zusammenfällt.\footnote{Die
  erste Untersuchung zum Einsatz von elektronischen Literaturdatenbanken
  stammt von Philipp Bagley (1951), der übrigens später ebenfalls den
  Begriff \enquote{Metadaten} prägte.} Durch eine einfache Suche konnten
weder \emph{Datenbanken von Datenbanken von Datenbanken} noch
\emph{Kataloge von Katalogen von Katalogen} gefunden werden (ihre
Existenz ist also nicht ausgeschlossen, aber zumindest zweifelhaft); es
gibt allerdings eine Reihe von \emph{Listen von Listen von Listen} sowie
Listen höherer Ordnung.\footnote{Beispielsweise die \emph{List of lists
  of lists} in der Englischsprachigen Wikipedia
  (\url{https://en.wikipedia.org/wiki/List_of_lists_of_lists}) und die
  irgendwann zwischen 2008 und 2011 von Scott Sisikind zusammengestellte
  \enquote{List of Lists of Lists of Lists}
  (\url{http://www.raikoth.net/lololol.html}).} Ob die Betrachtung solch
allgemeiner Meta-Sammlungen zur Erstellung von Sammlungen höherer
Ordnung zielführend ist, darf bezweifelt werden. Vielmehr macht es Sinn,
für weitergehende Untersuchungen den Akt des Sammeln und Beschreibens
anderer Sammlungen und Beschreibungen selbst zu betrachten.\footnote{Ein
  Ansatz für den Bereich von (Meta-)Daten liegt vor (Voß, 2013).}

\hyperdef{}{references}{\label{references}}
\section*{Literaturverzeichnis}\label{literaturverzeichnis}
\addcontentsline{toc}{section*}{Literaturverzeichnis}

\hyperdef{}{ref-Bagley1951}{\label{ref-Bagley1951}}
Bagley, P. R. (1951). Electronic Digital Machines for High-Speed
Information Searching.

\hyperdef{}{ref-Birch2015}{\label{ref-Birch2015}}
Birch, B. J. (2015). „Bibliographies of Bibliographies``, in \emph{The
Philatelic Bibliophile's Companion}, 1205--1206. URL:
\url{http://www.fipliterature.org/pbcompanion.PDF}.

\hyperdef{}{ref-Frank1978}{\label{ref-Frank1978}}
Frank, P. R. Hrsg. (1978). \emph{Von der systematischen Bibliographie
zur Dokumentation}. Darmstadt: Wissenschaftliche Buchgesellschaft.

\hyperdef{}{ref-Gorraiz2007}{\label{ref-Gorraiz2007}}
Gorraiz, J. (2007?). „Bibliographien der Bibliographien``, in
\emph{Bibliographie. Leitfaden zur konventionellen Bibliographie mit
besonderer Berücksichtigung von CD-ROM- und Web-Ressourcen} URL:
\url{http://homepage.univie.ac.at/juan.gorraiz/konven/bibbib.htm}.

\hyperdef{}{ref-Grundtvig1903}{\label{ref-Grundtvig1903}}
Grundtvig, V. (1903). Gedanken über Bibliographie. \emph{Centralblatt
für Bibliothekswesen} 20, 405--443.

\hyperdef{}{ref-Hjorland2008}{\label{ref-Hjorland2008}}
Hjørland, B. (2008). Information Literacy and Digital Literacy.
\emph{prisma.com}. URL:
\url{http://revistas.ua.pt/index.php/prismacom/article/view/684}.

\hyperdef{}{ref-Josephson1901}{\label{ref-Josephson1901}}
Josephson, A. (1901). \emph{Bibliographies of Bibliographies,
chronologically arranged, with occasional notes and an index}. Chicago:
Bibliographical Society of Chicago.

\hyperdef{}{ref-Josephson1905}{\label{ref-Josephson1905}}
Josephson, A. (1905). \emph{Proposition for the establishment of a
bibliographical institute}. Chicago.

\hyperdef{}{ref-Josephson1911a}{\label{ref-Josephson1911a}}
Josephson, A. (1911a). Bibliographies of Bibliographies. \emph{The
Bulletin of the Bibliographical Society of America} 3, 23--24.

\hyperdef{}{ref-Josephson1911b}{\label{ref-Josephson1911b}}
Josephson, A. (1911b). Bibliographies of Bibliographies. \emph{The
Bulletin of the Bibliographical Society of America} 3, 50--53.

\hyperdef{}{ref-Josephson1912}{\label{ref-Josephson1912}}
Josephson, A. (1912). Bibliographies of Bibliographies (Continuation).
\emph{The Bulletin of the Bibliographical Society of America} 4, 23--27.

\hyperdef{}{ref-Josephson1913b}{\label{ref-Josephson1913b}}
Josephson, A. (1912--1913a). Bibliographies of Bibliographies: Second
edition (concluded). \emph{The Papers of the Bibliographical Society of
America} 7, 115--123.

\hyperdef{}{ref-Josephson1913a}{\label{ref-Josephson1913a}}
Josephson, A. (1912--1913b). Bibliographies of Bibliographies: Second
edition (continued). \emph{The Papers of the Bibliographical Society of
America} 7, 33--34.

\hyperdef{}{ref-Keenan2015}{\label{ref-Keenan2015}}
Keenan, T. Hrsg. (2015). „Bibliographies of Bibliographies``, in
\emph{Russia, Eastern Europe, and Eurasia: A Research Guide} (Pricenton
University Library). URL:
\url{http://libguides.princeton.edu/c.php?g=84111\&p=545030}.

\hyperdef{}{ref-Lexoldie2007}{\label{ref-Lexoldie2007}}
Lexoldie, B. (2007). Bibliografie. \emph{Wikipedia}. URL:
\url{https://de.wikipedia.org/w/index.php?diff=prev\&oldid=27658711}.

\hyperdef{}{ref-Peignot1812}{\label{ref-Peignot1812}}
Peignot, G. (1812). \emph{Répertoire bibliographique universel}. Paris:
Chez Antoine-Augustin Renouard.

\hyperdef{}{ref-Petzholdt1866}{\label{ref-Petzholdt1866}}
Petzholdt, J. (1866). \emph{Bibliotheca bibliographica}. Leipzig:
Engelmann.

\hyperdef{}{ref-Roesch2003}{\label{ref-Roesch2003}}
Rösch, H., \& Härkönen, S. (2003). „Bibliographien der Bibliographien``,
in \emph{Allgemeine Informationsmittel} URL:
\url{http://www.fbi.fh-koeln.de/institut/personen/roesch/Material/_Roesch/Informationsmittel/Kapitel5.htm}.

\hyperdef{}{ref-Taylor1955}{\label{ref-Taylor1955}}
Taylor, A. (1955). \emph{A history of bibliographies of bibliographies}.
New Brunswick, N.J.: Scarecrow Press.

\hyperdef{}{ref-Voss2013}{\label{ref-Voss2013}}
Voß, J. (2013). Describing data patterns. A general deconstruction of
metadata standards.

%autor
\begin{center}\rule{0.5\linewidth}{\linethickness}\end{center}

\textbf{Jakob Voß} arbeitet im Bereich Forschung und Entwicklung an der
Verbundzentrale des GBV (VZG).

\end{document}